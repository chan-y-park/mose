\documentclass[11pt]{article}
\usepackage{color}
\usepackage{amsmath}
\usepackage{hyperref}
\usepackage{fullpage}
\usepackage{comment}
\usepackage{amsmath, amssymb}
\title{	The numerics of k-wall evolution}
\author{Chan, Daniel, Pietro}
%\date{}                                           % Activate to display a given date or no date
\newcommand{\be}{\begin{equation}}
\newcommand{\ee}{\end{equation}}
\newcommand{\IR}{\mathbb{R}}
\newcommand{\IC}{\mathbb{C}}

\usepackage{xcolor}
\hypersetup{
    colorlinks,
    linkcolor={blue!50!black},
    citecolor={green!50!black},
    urlcolor={blue!80!black}
}

\begin{document}
\maketitle
\tableofcontents

Efficient evolution is achieved by means of integration of Picard-Fuchs equations.
However, these are singular at the discriminant loci, where k-wall begin. Therefore a different technique must be adopted for early stage evolution of primary k-walls.

\section{Primary K-walls}
The basic equation governing k-walls is
\be
	arg(Z_{\gamma}(u)) \in e^{i\vartheta}\IR_{-} \qquad \Leftrightarrow \qquad du/dt \sim e^{i(\theta +\pi)} / \eta_{\gamma}(u)
\ee
where $\eta_{\gamma}(u):=dZ_{\gamma}(u) / du$.

In the vicinity of $u=u_{0}$ where $\gamma$ shrinks, two of the roots of the elliptic curve come close to each other
\be
	y^{2} = 4 x^{3} - g_{2}(u) x -g_{3}(u) = 4\prod_{i=1}^{3}(x-e_{i}(u))
\ee
in particular $e_{1}(u_{0})= e_{2}(u_{0})$.

As we move away, the period integral
\be
	\eta_{\gamma}(u) = \int_{\gamma} {dx\over y(x,u)} = 2 \int_{e_{1}(u)}^{e_{2}(u)}{dx\over y(x,u)}
\ee
can be evaluated by means of elliptic functions (see Gradshteyn-Rhyzhik formula ******).
\be
	\eta(u) = \frac{4}{\sqrt{e_{3}-e_{1}}} \, K\left( \frac{e_{2}-e_{1}}{e_{3}-e_{1}} \right)
\ee
note that in their conventions the argument of the K-function is the square-root of ours!
The above expression clearly has branch-cuts, due both to the square-root and to the K-function.
But near $u=u_{0}$ both pieces are well-behaved since (see Gradshteyn formula ***)
\be
	K(z) \sim {\pi\over 2}(1+ {1\over 4}z + {9\over 64} z^{2} +O(z^{3}))
\ee
Nevertheless, we cannot directly apply the formula fo $\eta$ to our differential equation, due to \emph{additional} branch-cuts, coming from the roots $e_{i}(u)$. These can be quite bad, as testified by the following plot \\
****** INSERT PLOT ********\\
But, having understood the \emph{nature} of branch-cuts, and their effects, we can suitably define evolution as follows.
We start with $u=u_{0}$, and compute $\eta_{0} = \eta(u_{0})$ there: this will be simply ($e_{3} = -e_{1}-e_{2} = -2e_{1}$)
\be
	\eta_{0} = \frac{4}{\sqrt{-3 e_{1}}}\frac{\pi}{2}.
\ee
Then we use this into our differential equation, and define
\be
	u_{1} = u_{0} + \delta t \cdot  e^{i(\theta +\pi)} / \eta_{0}
\ee
with small positive $\delta t$.
At $u=u_{1}$ we may now use our expression for $\eta$, but we must somehow be careful about branch-cuts. In particular, the ambiguity consists in picking $(e_{1}, e_{2})$ vs $(e_{2}, e_{1})$ to plug into $\eta$. Since there are only two possibilities, we compute them both
\be
	\eta_{12} \qquad \text{and} \eta_{21}
\ee
then, we use each of these to compute $du/dt$ at $u_{1}$, and we notice that only one of the two will give a tangent direction compatible with $u_{1}-u_{0}$. More precisely we choose the one giving the smaller value of
\be
	\left|arg\left(  \frac{e^{i(\theta +\pi)} / \eta_{12}}{u_{1}-u_{0}} \right) \right| %
	\qquad \text{vs} \qquad %
	\left|arg\left(  \frac{e^{i(\theta +\pi)} / \eta_{21}}{u_{1}-u_{0}} \right) \right|
\ee
this will tell us what is the correct value of $\eta_{u_{1}}$, thus fixing the ambiguity of branch-cuts.
Having $\eta(u_{1})$ allows also to compute 
\be
	\eta'(u_{1}) := \frac{\eta(u_{1}) - \eta(u_{0})}{u_{1}-u_{0}}\,.
\ee
We iterate this a few times, moving slowly away from the discriminant locus, until the distinction between $e_{3}$ and the other two roots becomes harder to make. At that point, we switch to Picard-Fuchs.

\bigskip

{\bf Possible substantial improvement of numerics:} instead of keeping track only of the 1st derivative, we could keep track numerically of corrections due to 2nd and higher order derivatives,
which should become accessible as we compute more and more points.


\section{Picard Fuchs}
For now, see old notes with Dima.

\end{document}  