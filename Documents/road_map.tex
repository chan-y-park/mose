\documentclass[11pt]{report}
\usepackage{color}
\usepackage{hyperref}
\usepackage{fullpage}
\usepackage{amssymb}
\title{A rough road map for the \emph{MOSE} project}
\author{Chan, Daniel, Pietro}
%\date{}                                           % Activate to display a given date or no date

\usepackage{xcolor}
\hypersetup{
    colorlinks,
    linkcolor={blue!50!black},
    citecolor={green!50!black},
    urlcolor={blue!80!black}
}
\definecolor{lightgray}{gray}{0.80}

\begin{document}
\maketitle

\tableofcontents


\chapter{K wall networks}

\section{Multithreading - {\color{red} \bf done}}
At least the generation of each K-wall network should be done in a single thread, and multithreading should be implemented such that during {\tt phase\_scan()} we can run a multi-threaded program.

Saving to and loading from files will take a long time, and running it on a separate thread will be beneficial, although its implementation will not be straightforward.

It is worthwhile to consider how to implement a finer threading, for example a single thread for each K-wall, but this may be too fine and the overhead of threading management will be overwhelming. But on the other hand a thread per network seems to be too crude.

\section{Intersection algorithm  - {\color{red} \bf done}}
{\bf Warning:} current algorithm assumes that there is a single intersection between two segments in the same bin.

This should be a function called \texttt{find\_intersections(trajectory\_1,trajectory\_2)} returning the intersection(s) of the two given trajectories. It shold be placed in the numerics.py file\\
If the trajectories don't intersect, it should return {\tt []} otherwise it should return a list {\tt [[point, index\_1, index\_2], ...]} where
\begin{itemize}
\item {\tt point} is the exact point of intersection, given as a complex number 
\item {\tt index\_1} is the index of the element of {\tt trajectory\_1.coordinates} that corresponds to the exact intersection point. (Same for {\tt index\_2}). This is important because it will be used to determine the pertinent charge to use in computing progenie by means of the KSWCF. {\bf We are taking the shortcut of \emph{not} adding the actual point to the trajectory, so we introduce a tiny approximation. Keep in mind if trouble arises.}
\item The exact point of intersection should be added to the lists {\tt trajectory\_1.coordinates} and {\tt trajectory\_2.coordinates}, at the appropriate point {\bf warning:} coordinates attributes in the Trajectory class are given as a 2-vector, not as a complex numer. 
\item Moreover, the corresponding value of the period of $\eta$ for that trajectory should also be added to {\tt trajectory\_1.periods} (and same for trajectory\_2), this goes as a complex number.
\end{itemize}

\section{KSWCF}
\subsection{Basics - {\color{red} \bf done} {\tiny \bf (maybe can improve a bit more)}}
This should be a function called {\tt progeny\_2(data)} placed in the file kswcf.py, that takes {\tt data = [ [ $\gamma_{1}$, $\Omega(\gamma_{1})$  ] ,[$\gamma_{2}$, $\Omega(\gamma_{2})$] ]} and returns a list of new charges and corresponding degeneracies, \emph{excluding the parents}:
{\tt [ ...  [ $n \gamma_{1}+ m \gamma_{2}$ , $\Omega(n \gamma_{1}+ m \gamma_{2})$  ]  ... ]}.
A primitive example is provided in the file kswcf.py.

\subsection{Enhancement}
At intersections, make KSWCF really handled by phases of central charges (right now it's auto-correcting to get positive pairing). This can be done once signs of priary k-walls are fixed by weierstrass.


\section{Rotation of the phase - {\color{red} \bf done}}
A function called {\tt phase\_scan(theta\_range)} in structure.py, that repeats the whole computation of primary k-walls, their intersections, generation of descendants, and iterations to get new intersections, new descendats etc.. for different phases.\\
The argument {\tt theta\_range} is of the form {\tt theta\_range = [theta\_in, theta\_fin, steps]}.

At each phase, it \emph{stores the information about intersection points and their genealogy}. This information will then be used to construct full MS walls.

As of now, it keeps track of all intersections and all k-walls at every phase, and returns a big array containing them: this has the shape {\tt [  all\_ints , all\_trajs  ]} where {\tt all\_ints = [ ... [ int1, int2, ... ] ... ]} contains arrays of intersections of given phase and similarly {\tt all\_trajs = [ ... [ traj1, traj2, ... ] ... ]} contains arrays of trajectories of given phase.



\section{Genealogy - {\color{red} \bf done}}
Not sure how to structure this yet. But it should be a function called {\tt build\_genealogy\_tree(intersection\_point)} that takes as argument an object of the class IntersectionPoint.
It should use the attribute {\tt intersection\_point.parents} to recursively determine the trajectories from whom the point descends.
It should eventually return a structure (type of it is to be determined) that contains the genealogy of such point, meaning which branch-points are the farthest ancestors of the intersection, and in which associative order, eg: {\tt [[BP1,BP2],BP3]}.
Not sure if keeping track of branch-point ancestors will be enough, maybe not. Maybe should keep track of trajectories in the middle as well.

\medskip

The purpose of genealogy is to identify which intersections (occurring at different phases) belong to the same wall of marginal stability.


\section{Construction of MS walls  - {\color{red} \bf done}}
Once the phase-rotation algorithm is working and the genealogy is ready, we should be able to group together intersection points coming from different phase-snapshots, and build analytic MS walls out of them. This will need some thinking


\section{Branch cuts - {\color{red} \bf done}}
{\bf \color{blue} \large Note: done in a different way than suggested here. The main difference is that I got rid of branch-cuts, and only used branch-points, taking all cuts to extend vertically to infinity.}\\
In the function {\tt prepare\_branch\_locus} in file structure.py (\emph{warning: the whole structure of the code has undergone a major change, need to look for this function}), should create an array of branch cuts called {\tt bcts} that is to be returned (see the function there).
Each branch point will generate a branch cut. Cuts should be straight lines going to infinity, at a certain angle {\tt THETA\_CUTS} to be specified in the file {config.py}. A tentative implementation of this kind of cuts is already provided by the {\tt init} method of the {\tt BranchCut} class.
It remains to
\begin{itemize}
\item tune the parameter {\tt branch\_cut\_cutoff} in file parameters.py (length of the branch cut)
\item write the method {\tt check\_{cuts}} in the class {\tt Trajectory} (\emph{Warning: now this class has been replaced by the {\tt KWall} class and its daughter classes}), which is supposed to {\bf detect intersection of trajectories with cuts}, and determine the {\tt Trajectory.splittings} (find exact point, add it to {\tt Trajectory.coordinates}, then the splitting contains the indices of these exact points. The corresponding kind of addition must be done to {\tt Trajectory.periods}) and determine the values of {\tt Trajectory.local\_{charge}} on segments of the trajectory between the various {\tt splittings}. See the dummy method in the Trajectory class.
\item Update the intersection-searching procedure: when two K-walls intersect, the intersection pairing should be evaluated based on the charge of the K-wall at the intersection, this of course can only happen after K-walls have evolved \emph{and} intersections with cuts have been taken into account.
\end{itemize}


\section{Branch-cut crossing (possible) enhancement}
When computing branch-cuts crossing of K-walls, should add the intersection points to the trajectory? 
Right now it's just taking the nearest point to handle the intersection.


\section{Charges of branch points (Weierstrass module)}
This should be an algorithm that, at the very beginning ---while creating the branch locus--- determines automatically which charges to assign to each branch point (hence to each branch cut).
It should also specify the DSZ matrix. All this information should replace {\tt DSZ\_MATRIX} and {\tt FIXED\_CHARGES} which are now given by hand in the config.py file.


\subsection{interfacing of main code with weierstrass module}
\begin{itemize}
\item mainly the weierstrass module will replace the elliptic\_fibration module
\item Branch-points have a new attribute called 'monodromy\_matrix' which should be determined by Weierstrass
\item around line 30 of k\_wall\_network.py, the sign = [-1,+1] or [+1,-1] should also be determined by Weierstrass
\end{itemize}


\section{Storing Data - {\color{red} \bf done -- see instructions}}
When running a long computation, involving probing several (such as O(100)) phases, it would be good to store all data on a separate file.
In particular, we should 
\begin{itemize}
	\item[\checkmark] Create an external file(s) whose name contains date and time
	\item[\checkmark] Store data in such a way that it is retrievable: just need to store the output of the function {\texttt{phase\_scan}}
	\item[\checkmark] Write a module for retrieving (reloading in the RAM) the above data ready for analysis
	\item[\checkmark] Bonus: write a module that (using the external data) will create a separate folder, containing snapshots of the k-walls on the moduli space for all scanned phases
\end{itemize}

\medskip


{\color{red} \bf \Large\underline{Instructions}}

To save data, either a single k-wall network, or an entire phase-scan and the related MS walls, use the option {\tt -w} in the command line. 
For example:

\smallskip

%\indent \colorbox{lightgray}{\tt \$ python2.7 \_\_main\_\_.py -l info -s 0 -w } $\qquad$ (e.g. in bash)
\indent \colorbox{lightgray}{\tt \$ python2.7 -m mose -l info -s 0 -w -g -c fibration\_{invented}.ini} 

\hspace{0.3\textwidth}{(e.g. in bash, form the folder containing {\tt mose/})}

\smallskip

or

\smallskip

\indent \colorbox{lightgray}{\tt run \_\_main\_\_  -l info -s 0 -w -g -c fibration\_{invented}.ini} $\qquad\qquad\qquad$ (in ipython)

\smallskip


This will produce a file, whose name will be of the type:
\begin{center}
{\tt label\_YYYY-MM-DD-hh.mm.mose} 
\end{center}
with {\tt label} being either
\begin{center}
{\tt single\_network} $\quad$ or $\quad$  {\tt phase\_scan} .
\end{center}

To read the data, from the python shell import the function {\tt f\_{recover}} from the file {\tt save\_to\_file.py}, i.e. use:

\smallskip

\indent \colorbox{lightgray}{\tt>>> from save\_to\_file.py import f\_recover}

\smallskip

\noindent Then use this function to load the data from a {\tt *.mose} file:

\smallskip

\indent \colorbox{lightgray}{\tt>>> f\_recover(`some\_file\_of\_type.mose')}

\bigskip

\noindent Now there are two possibilities: 
\begin{itemize}
	\item If the {\tt label} is {\tt single\_network} then the above command return an instance of the class {\tt KWallNetwork}.
	\item If the  {\tt label} is {\tt phase\_scan} then the above command will return a list of the form\\
	\begin{center}  {\tt [[KWN${}_{1}$, KWN${}_{2}$, $\dots$, KWN${}_{m}$], [MS${}_{1}$, \dots MS${}_{\ell}$]]}  \end{center}
	with {\tt KWN, MS} being instances respectively of the classes {\tt KWallNetwork, MarginalStabilityWall}
\end{itemize}

In either case, you can the go on and load plotting functions from our libraries and analyze the data.


\section{Refined numerical issues with evolution  - {\color{red} \bf done}}
Trajectories evolve according to the PF equations, these are governed by a matrix which becomes singular at the singular locus of the moduli space (since they contain the discriminant).
The {\texttt{odeint}} algorithm will get into trouble integrating PF in the neighborhood of these points.
In particular, the problems seem to consist in trajectories bending abruptly when they run close to singularities.

\medskip

\noindent Possible solutions:
\begin{itemize}
	\item Once a trajectory was evolved, search backwards to see if it passed near a singularity. If it did, cut it in a neighborhood of the singular point (where it enters a disk of radius $\epsilon$ centered there), then invoke a new evolution algorithm (such as that of primay k-walls), then revert back to PF.
	\item Once a trajectory was evolved, search backwards to see if it passed near a singularity. If it did, erase the trajectory before next iteration, so it cannot create bogus offsprings due to bad numerics (e.g. the bending will make it intersect k-walls it wasn't supposed to meet).
\end{itemize}

The first possibility seems better, however the evolution method of primary walls will likely run into trouble too, recall the behavior of that method near a MS wall (which certainly would be met by the evolving kwall, when crossing the disc centered around the singular point).


The second possibility might be numerically heavy, but it could be greatly improved by tracking e.g. the determinant of the PF matrix, and using that as a red flag, instead of computing distances from singularities. It would clearly leave an unexplored area nearby a singularity, however this can be fixed by the idea of section \ref{sec:bp-to-int}

\bigskip

{\color{blue} \noindent{\bf Note: } solved with the second option. Introduced a parameter called {\tt{TRAJECTORY\_SINGULARITY\_THRESHOLD}} that controls the value of the {\bf determinant of the picard-fuchs matrix} along the trajectory's evolution, above which a trajectory is deemed too close to a singularity. To start with, this threshold is set to be $10^{6}$.}



\section{Enhancing MS walls}\label{sec:bp-to-int}
\subsection{Adding endpoints  - {\color{red} \bf done}}
A very simple thing to do: from the genealogy data of an MS wall it should be possible to tell whether the wall starts/ends at certain singularities. Then, make a module that creates new intersection-like objects from data of those singularities, and add these to the MS wall. 

\subsection{Interpolating function}
Replace numerical data points with anapproximate interpolating function to have a nice smooth MS-wall. A simpler alternative is just to draw a line matching all the dots, although it won't be as nice.




\section{Fibration-handling module}
\subsection{Storing and calling fibrations - {\color{red} \bf done}}
We have a code that handles a class of fibrations. The separate data specifying each fibration should be stored in a tidy way, and called by the core part of the program with a simple function.
A possibility is to have a different text file containing the data of each fibration, formatted in a standard way, and to load such data by giving the name of the file (e.g. {\tt SU2\_Nf=0.xml}).

\subsection{Creating arbitrary fibrations}
A nice addition would be a function which allows the user to add a custom-made fibration. For example, one could be prompted to specify the number of UV parameters $m,\Lambda,\dots$, then be prompted for $g_{2}(u), g_{3}(u)$. Then the code would store this data in one of the fibration files described above, permanently, with a user-defined name. Then this fibration would be available to the user for future study.



\section{Tilting the u-plane}
Make sure that branch cuts do not overlap. If two branch-points align vertically, the code should automatically rotate the u-plane by a phase since the beginning.
A suitable phase can be determined automatically to be e.g. a half of the smallest phase between any two discriminant loci.



\section{Visualization module}
As we will be analyzing a big array of data, and storing it for later analysis, it is desirable to have a separate suite that handles visualization of this formatted data.
This will have a close interplay with the GUI. The GUI will pass to this module certain data such as KWallNetwork or MSWall type of objects.\\
So this module should provide (probably) \emph{two} functions: one for single K-wall-network frames, and the other for MS walls and multiple K-wall-networks. 
It should take some \emph{options} to determine what kind of data to print 
\begin{itemize}
\item charges of branch points / Kwalls at various points, 
\item degeneracies of Kwalls, 
\item MS wall data such as charges and pairing
\item ...
\end{itemize}




\section{GUI}
Write a graphical user interface. Basic functionalities:
\begin{itemize}
	\item Analyzing both single networks and phase-scans with MS walls, storing the data in an external file
	\item Loading the data stored in {\tt*.mose} files and analyzing it, e.g. by plotting it, or by printing certain relevant data, such as charges, degeenracies, etc etc.
	\item Bonus: write a module that uses the screenshots produced by the function scanning multiple networks, to provide a window with a slider tool and lets you browse the k-wall evolution. This is quite useful for debugging and checking the numerics (esp. the undesired bending of k-wall at critical phases)
\end{itemize}



\section{MOSE distribution packages}
Figure out how to create installation packages for major OS's (are tablets fair game as well?).


\chapter{Extracting Refined Topological Invariants}

For the distant future: figure out how to handle ``finite networks'', and automatically recover topological information on the fibration.





\chapter{Hyperkahler metrics}

For the distant future: figure alorithms for solving TBA equations and getting the explicit metrics.




\end{document}  