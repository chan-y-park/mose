\documentclass[11pt]{article}
\usepackage{color}
\usepackage{amsmath}
\usepackage{hyperref}
\usepackage{fullpage}
\usepackage{comment}

\def\CN{{\mathcal{N}}}

\title{Some notes on the fibrations}
\author{Chan, Daniel, Pietro}
%\date{}                                           % Activate to display a given date or no date
\newcommand{\be}{\begin{equation}}
\newcommand{\ee}{\end{equation}}


\usepackage{xcolor}
\hypersetup{
    colorlinks,
    linkcolor={blue!50!black},
    citecolor={green!50!black},
    urlcolor={blue!80!black}
}

\begin{document}
\maketitle
\begin{center}
	Some notes on the conventions used for fibration data within our code.
\end{center}
\tableofcontents


\section{$SU(2)$ $N_{f}=1$}

From section 2 of 9706145 (Bilal-Ferrari)
\be
	y^{2} = x^{2}(x-u) + {m \Lambda^{3}\over 4} x - {\Lambda^{6}\over 64}
\ee
shifting $z = x + u/3$ brings this into Weierstrass normal form
\be
\begin{split}
	y^{2} & = z^{3} + f(u) z + g(u) \\
	f(u) & = {\Lambda ^3 m\over 4} - \frac{u^2}{3}\\
	g(u) & = -\frac{\Lambda ^6}{64}+\frac{ \Lambda ^3 m u}{12} -\frac{2 u^3}{27}
\end{split}
\ee
which is the form appearing in the code.

\begin{comment}
From section 10.2 of GMN2
\be
	\lambda^{2} = \left( {\Lambda^{2} \over z^{3}} + {3 u\over z^{2}} + {2 m \Lambda^{} \over z^{}} + \Lambda^{2} \right)\, dz^{2}
\ee
upon a proper ($z$-dependent) rescaling od $\Lambda$, we may identify the elliptic curve
\be
	y^{2} = 4z^{3} + {8m\over \Lambda}z^{2}+{12 u\over \Lambda^{2}} z + 4
\ee
shifting $z \to z - 2m / 3\Lambda$ brings this into Weierstrass normal form
\be
\begin{split}
	y^{2} & = 4z^{3} - g_{2}(u) z - g_{3}(u) \\
	g_{2}(u) & = {16 m^{2} \over 3\Lambda^{2}} - {12 u \over \Lambda^{2}} \\
	g_{3}(u) & = {8 m u \over \Lambda^{3}} - {64 m^{3} \over 27 \Lambda^{3}} 
\end{split}
\ee
which is the form appearing in the code.
\end{comment}




\section{$SU(2)$ $N_{f}=2$}

From section 2 of 9706145 (Bilal-Ferrari)
\be
	y^{2} = x^{2}(x-u) - \frac{\Lambda^{4}}{64}(x-u) + \frac{\Lambda^{2}}{4}m_{1}m_{2}x - \frac{\Lambda^{4}}{64}(m_{1}^{2}+m_{2}^{2})
\ee
shifting $z = x + u/3$ brings this into Weierstrass normal form
\be
\begin{split}
	y^{2} & = z^{3} + f(u) z + g(u) \\
	f(u) & = -\frac{\Lambda ^4}{64}+\frac{1}{4} \Lambda ^2 m_1 m_2-\frac{u^2}{3}\\
	g(u) & = \frac{1}{12} \Lambda ^2 m_1 m_2 u-\frac{2 u^3}{27}+\frac{\Lambda ^4 u}{96} -\frac{1}{64} \Lambda ^4 \left(m_1^2+m_2^2\right)
\end{split}
\ee





\section{$SU(2)$ $N_{f}=3$}

From section 2 of 9706145 (Bilal-Ferrari)
\be
\begin{split}
	y^{2} &= x^{2}(x-u) - \frac{\Lambda^{2}}{64}(x-u)^{2} -  \frac{\Lambda^{2}}{64}(x-u)(m_{1}^{2}+m_{2}^{2}+m_{3}^{2})  \\
	& + \frac{\Lambda}{4}m_{1}m_{2}m_{3}x - \frac{\Lambda^{2}}{64}(m_{1}^{2}m_{2}^{2} + m_{2}^{2}m_{3}^{2} + m_{3}^{2}m_{1}^{2})
\end{split}
\ee
shifting $z = x + u/3$ brings this into Weierstrass normal form
\be
\begin{split}
	y^{2} & = z^{3} + f(u) z + g(u) \\
	f(u) & = -\frac{\Lambda ^4}{12288}-\frac{1}{64} \Lambda ^2 \left(m_1^2+m_2^2+m_3^2\right)+\frac{1}{4} \Lambda  m_1 m_2 m_3-\frac{u^2}{3}+\frac{\Lambda ^2 u}{48} \\
	g(u) & = -\frac{1}{27} \left(2 u^3\right)-\frac{5 \Lambda ^2 u^2}{576}+\frac{\Lambda ^4 u}{9216}-\frac{\Lambda ^6}{3538944}+\frac{1}{12} \Lambda  m_1 m_2 m_3 u+\frac{1}{768} \Lambda ^3 m_1 m_2 m_3\\
	 & +\frac{1}{96} \Lambda ^2 \left(m_1^2+m_2^2+m_3^2\right) u-\frac{\Lambda ^4 \left(m_1^2+m_2^2+m_3^2\right)}{12288}-\frac{1}{64} \Lambda ^2 \left(m_2^2 m_1^2+ m_{3}^{2}m_1^2+m_2^2 m_3^2\right)
\end{split}
\ee




\section{$SU(2)$ $\CN=2^{*}$}
The theory has two UV parameters: $\tau, m$.
The elliptic curve is given in equation (16.24) of SW-2
\be
	y^{2} = \prod_{i=1}^{3}\left(  x - e_{i} \widetilde u -{1\over 4} e_{i}^{2} m^{2}  \right)
\ee
where a $m,\tau$-dependent translation of the $u$-plane has been employed
\be
	\widetilde u = u - {1\over 8} e_{1} m^{2}\,.
\ee
To write this in Weierstrass form, we make a translation on the $x$-plane by taking
\be
	x\quad \to\quad x + \frac{1}{12} \left( m^{2} \, (e_{1}^{2} +e_{2}^{2} + e_{3}^{2}) + 4  \widetilde{u} (e_{1}+e_{2}+e_{3})\right) \equiv x + \frac{1}{12}  m^{2} \, (e_{1}^{2} +e_{2}^{2} + e_{3}^{2})
\ee
and we employ the Weierstrass invariants
\be
	e_{1}e_{2}+e_{2}e_{3}+e_{3}e_{1} = -{g_{2}\over 4}\qquad e_{1}e_{2}e_{3} = {g_{3}\over 4}
\ee 
overall we get the simple formula
\be
\begin{split}
	y^{2} &=  x^{3} + \frac{1}{1728} \big(-\frac{9}{4} g_2^2 m^4-324 g_3 m^2 \widetilde{u}-432 g_2 \widetilde{u}^2\big) x\\
	& +\frac{1}{1728}\left(\frac{1}{32} g_2^3 m^6-\frac{27}{16} g_3^2 m^6-\frac{27}{4} g_2 g_3 m^4 \widetilde{u}-18 g_2^2 m^2 \widetilde{u}^2-432 g_3 \widetilde{u}^3\right)
\end{split}
\ee
\emph{It would be nice to go ahead and express this in terms of $\tau$ directly, using the expressions of $g_{2}, g_{3}$ in terms of Eisenstein series, but on the other hand, any choice of $g_{2}, g_{3}$ does correspond to a $\tau$, so we will always be in the allowed parameter space, although we need to do some work to obtain the value of $\tau$ from the invariants.}




\end{document}  