\documentclass[11pt]{article}
\usepackage{color}
\usepackage{amsmath}
\usepackage{hyperref}
\usepackage{fullpage}
\usepackage{comment}
\title{Some notes on the fibrations}
\author{Chan, Daniel, Pietro}
%\date{}                                           % Activate to display a given date or no date
\newcommand{\be}{\begin{equation}}
\newcommand{\ee}{\end{equation}}


\usepackage{xcolor}
\hypersetup{
    colorlinks,
    linkcolor={blue!50!black},
    citecolor={green!50!black},
    urlcolor={blue!80!black}
}

\begin{document}
\maketitle
\begin{center}
	Some notes on the conventions used for fibration data within our code.
\end{center}
\tableofcontents

\section{Conventions!}
We work with the following form of elliptic fibrations:
\be
\begin{split}
	y^{2} & = 4z^{3} - g_{2}(u) z - g_{3}(u) 
\end{split}
\ee
with
\be
\begin{split}
	\Delta(u) & = (g_{2}(u))^{3} - 27 (g_{3}(u))^{2}\,.
\end{split}
\ee


\section{$SU(2)$ $N_{f}=1$}

From section 2 of 9706145 (Bilal-Ferrari)
\be
	y^{2} = x^{2}(x-u) + {m \Lambda^{3}\over 4} x - {\Lambda^{6}\over 64}
\ee
rescaling $y$ by a factor of $2$, and shifting $z \to z + u/3$ brings this into Weierstrass normal form
\be
\begin{split}
	y^{2} & = 4z^{3} - g_{2}(u) z - g_{3}(u) \\
	g_{2}(u) & = -\Lambda ^3 m+\frac{4 u^2}{3}\\
	g_{3}(u) & = \frac{\Lambda ^6}{16}-\frac{ \Lambda ^3 m u}{3} +\frac{8 u^3}{27}
\end{split}
\ee
which is the form appearing in the code.

\begin{comment}
From section 10.2 of GMN2
\be
	\lambda^{2} = \left( {\Lambda^{2} \over z^{3}} + {3 u\over z^{2}} + {2 m \Lambda^{} \over z^{}} + \Lambda^{2} \right)\, dz^{2}
\ee
upon a proper ($z$-dependent) rescaling od $\Lambda$, we may identify the elliptic curve
\be
	y^{2} = 4z^{3} + {8m\over \Lambda}z^{2}+{12 u\over \Lambda^{2}} z + 4
\ee
shifting $z \to z - 2m / 3\Lambda$ brings this into Weierstrass normal form
\be
\begin{split}
	y^{2} & = 4z^{3} - g_{2}(u) z - g_{3}(u) \\
	g_{2}(u) & = {16 m^{2} \over 3\Lambda^{2}} - {12 u \over \Lambda^{2}} \\
	g_{3}(u) & = {8 m u \over \Lambda^{3}} - {64 m^{3} \over 27 \Lambda^{3}} 
\end{split}
\ee
which is the form appearing in the code.
\end{comment}


\end{document}  